% The definition of Mit (self-contained paper)
%
% Written by Reuben Thomas 1993-2020.
% This file is in the public domain.

\documentclass[a4paper]{article}
\usepackage[british]{babel}
\usepackage[utf8]{inputenc}
\usepackage{newpxtext,booktabs,hyperref,bytefield,pbox,amsmath}
\frenchspacing


% Macros for this document

% Font for stack pictures; macro \spic includes italic correction
\newcommand{\spic}[1]{\texttt{\slshape{#1\/}}}

% Indefinite numbers of stack items
\newcommand{\x}[1]{\spic{x$_{#1}$}}

% Lay out an instruction definition
% Define the widths of the stack effect and description columns
\newlength{\itemwidth}\itemwidth=\textwidth \advance\itemwidth by -0.1in
\newlength{\instname}\instname=0.8in
\newlength{\stackcom}\stackcom=3.7in

\newcommand{\instprim}[3]{\item[]\parbox{\itemwidth}%
{\makebox[\instname][l]{\tt #1}%
\makebox[\stackcom][r]{\spic{\pbox[t]{\stackcom}{#2}}}\\[0.5ex]#3}}
\newcommand{\inst}[4]{\instprim{#1}{\tt …#2 → …#3}{#4}}

% Tabular hex numerals
% From https://tex.stackexchange.com/questions/543464/tabular-hex-numerals/543466#543466
\usepackage{tokcycle}
\newsavebox\lettwd
\savebox\lettwd{{\ooalign{0\cr a\cr b\cr c\cr d\cr e\cr f}}}
\tokcycleenvironment\tblhex
  {\addcytoks{\makebox[\wd\lettwd]{##1}}}
  {\processtoks{##1}}
  {\addcytoks{##1}}
  {\addcytoks{\makebox[\wd\lettwd]{##1}}}

% Lay out a line of the opcode table
\newcommand{\opcodetblone}[2]{0x#1 & {\tt #2} \\}
\newcommand{\opcodetbltwo}[4]{0x\tblhex#1\endtblhex & {\tt #2} & \tblhex0x#3\endtblhex & {\tt #4} \\}

% Label a line of the instruction encoding
\newcommand{\insttbllabel}[1]{\hspace{2ex}\bitbox[]{4}{\raggedright #1}}


\title{Mit virtual machine specification}
\author{Reuben Thomas}
\date{22nd July 2020}

\begin{document}
\maketitle

\section{Introduction}

Mit is a simple virtual machine, designed to be easy both to implement efficiently on most widely-used hardware, and to compile for. It aims to be formally specified. This specification is intended for those who wish to implement or program Mit.


\section{Parameters}

The virtual machine has the following parameters:

\begin{center}
\begin{tabular}{cp{3.25in}} \toprule
Endianness & Memory can be either big- or little-endian. \\
{\tt word\_bytes} & The number of bytes in a word, $4$ or $8$. \\
 \bottomrule
\end{tabular}
\end{center}


\section{Memory}

The flat linear address space contains {\tt word\_bytes}-byte words of $8$-bit bytes. Addresses are unsigned words and identify a byte; the address of a quantity larger than a byte is that of the byte in it with the lowest address. Whether a given word may be read or written can change during execution.


\section{Registers}
\label{registers}

The registers are word quantities:

\begin{center}
\begin{tabular}{cp{3.75in}} \toprule
\bf Register & \bf Function \\
 \midrule
{\tt pc} & The {\tt p}rogram {\tt c}ounter. Points to the next word of code. \\
{\tt ir} & The {\tt i}nstruction {\tt r}egister. Contains instructions to be executed. \\
 \bottomrule
\end{tabular}
\end{center}


\section{Stack}

Computation is performed with series of nested last in, first out stacks. The {\bf catch stack} is a stack of call stacks; a {\bf call stack} is a stack of computation stacks; and a {\bf computation stack} is a stack of words. To {\bf push} an item on to a stack means to add a new item to the top; to {\bf pop} an item means to remove the top item.

The stack of each sort implicitly used by instructions at a given moment is called the {\bf current} stack; the current stack is normally the top-most. Where it is not ambiguous, we refer simply to “the stack”.

Instructions implicitly pop their arguments from and push their results to the current computation stack. Subroutine call and return respectively push and pop a computation stack to and from the call stack; error handling call ({\tt catch}) pushes a call stack to the catch stack, and raising an error ({\tt throw}) pops a call stack from the catch stack.

The effect of an instruction on the catch stack is shown by its {\bf stack effect}:

\begin{center}
  {\tt \spic{before → after}}
\end{center}

\noindent where \spic{before} and \spic{after} are stack pictures showing the items on top of the stack before and after the instruction is executed. An instruction only affects the items shown in its stack effect.

A {\bf stack picture} represents the catch stack. Ellipsis is used to elide parts of the stack. A computation stack picture looks like this:

\begin{center}
\spic{…i$_n$\dots i$_1$}
\end{center}

\noindent Each \spic{i$_k$} is a stack item, and \spic{i$_1$} is on top of the stack. A stack item \spic{[x]} in square brackets is optional.

Computation stacks in a call stack are shown separated by a vertical bar {\rm |}. A caller–callee pair of adjacent computation stacks is shown thus:

\begin{center}
\spic{…j$_m$\dots j$_1$ {\rm |} …i$_n$\dots i$_1$}
\end{center}

\noindent where the \spic{j$_k$} are on the caller’s computation stack and the \spic{i$_k$} on the callee’s stack.

Call stacks in a catch stack are shown separated by a double vertical bar {\rm ‖}. A catcher–catchee pair of adjacent call stacks is shown thus:

\begin{center}
\spic{…j$_m$\dots j$_1$ {\rm ‖} …i$_n$\dots i$_1$}
\end{center}

\noindent where the \spic{j$_k$} are on the catcher’s top-most computation stack and the \spic{i$_k$} on the catchee’s top-most stack. Note that the ellipsis at the left-hand end of each computation stack may therefore elide whole call stacks.


\section{Execution}
\label{execution}

Execution proceeds as follows:

\begin{tabbing}
\hspace{0.2in}\=Rep\=eat:\+\+ \\*
Let \textbf{opcode} be the least significant byte of {\tt ir}. \\*
Shift {\tt ir} arithmetically one byte to the right. \\*
Execute the instruction given by \textbf{opcode}, \\*
\hspace{1em}or throw error $-1$ if the opcode is invalid. \-
\end{tabbing}

\noindent {\bf Instruction fetch} is the action of setting {\tt ir} to the word pointed to by {\tt pc} and making {\tt pc} point to the next word. This is done whenever extra instruction $0$ (see section~\ref{extra}) or trap $-1$ (see section~\ref{trap}) is executed.


\subsection{Errors and termination}
\label{errors}

In exceptional situations, such as an invalid memory access or division by zero, an {\bf error} may be {\bf thrown}; see section~\ref{control}. A {\bf status code} is returned to the catcher.

Execution can be terminated explicitly by a {\tt throw} instruction (see section~\ref{control}), which throws an error.

Status codes are signed numbers. $0$ to $-127$ are reserved for the specification. The defined status codes are given in table~\ref{statustable}.

\begin{table}[htb]
\begin{center}
\begin{tabular}{rp{4in}} \toprule
\bf Code & \bf Meaning \\ \midrule
$0$ & Normal termination. \\
$-1$ & Invalid opcode (see section~\ref{encoding}). \\
$-2$ & Stack overflow. \\
$-3$ & Invalid stack read. \\
$-4$ & Invalid stack write. \\
$-5$ & Invalid memory read. \\
$-6$ & Invalid memory write. \\
$-7$ & Address is valid but insufficiently aligned. \\
$-8$ & Division by zero attempted (see section~\ref{arithmetic}). \\
$-9$ & Division overflow (see section~\ref{arithmetic}). \\
 \bottomrule
\end{tabular}
\caption{\label{statustable}Status codes}
\end{center}
\end{table}

\noindent Errors $-2$ to $-9$ inclusive are optional: an implementation may choose not to detect these conditions.


\section{Instructions}

The instructions are listed below, grouped according to function, in the following format:

\begin{description}
\inst{NAME}{before}{after}{Description.}
\end{description}

\noindent The first line consists of the instruction’s name on the left, and its stack effect on the right; underneath is its description.

Numbers are represented in two’s complement form. Where a stack item’s name (including any numerical suffix) appears more than once in a stack effect, it refers each time to the identical stack item. Ellipsis is used for indeterminate numbers of items. An item called \spic{count} is interpreted as an unsigned number.


\subsection{Stack manipulation}

These instructions manage the computation stack.

\nopagebreak
\begin{description}
\inst{pop}{x}{}{Pop \spic{x} from the stack.}
\inst{dup}{\x{\spic{count}}\dots\x0 count}{\x{\spic{count}}\dots\x0 \x{\spic{count}}}{Pop \spic{count}. Copy \x{\spic{count}} to the top of the stack.}
\inst{set}{\x{\spic{count}+1}\dots\x0 count}{\x0 \x{\spic{count}}\dots\x1}{Set the \spic{count}$+1$th stack word to \spic{\x0}, then pop \x0.}
\inst{swap}{\x{\spic{count}+1}\dots\x0 count}{\x0 \x{\spic{count}}\dots\x1 \x{\spic{count}+1}}{Exchange the top stack word with the \spic{count}$+1$th.}
\end{description}


\subsection{Memory}

These instructions fetch and store quantities to and from memory.

\begin{description}
\inst{load}{addr}{val}{Push the word \spic{val} stored at \spic{addr}, which must be word-aligned.}
\inst{store}{val addr}{}{Store \spic{val} at \spic{addr}, which must be word-aligned.}
\inst{load1}{addr}{val}{Push the byte stored at \spic{addr}, setting unused high-order bits to zero, giving \spic{val}.}
\inst{store1}{val addr}{}{Store the least-significant byte of \spic{val} at \spic{addr}.}
\inst{load2}{addr}{val}{Push the $2$-byte quantity stored at \spic{addr}, setting unused high-order bits to zero, giving \spic{val}. \spic{addr} must be a multiple of $2$.}
\inst{store2}{val addr}{}{Store the $2$ least-significant bytes of \spic{val} at \spic{addr}, which must be a multiple of $2$.}
\inst{load4}{addr}{val}{Push the $4$-byte quantity stored at \spic{addr}, setting any unused high-order bits to zero, giving \spic{val}. \spic{addr} must be a multiple of $4$.}
\inst{store4}{val addr}{}{Store the $4$ least-significant bytes of \spic{val} at \spic{addr}, which must be a multiple of $4$.}
\end{description}


\subsection{Constants}

\begin{description}
\inst{push}{}{val}{Push the word \spic{val} stored at {\tt pc}, and add {\tt word\_bytes} to {\tt pc}.}
\inst{pushrel}{}{addr}{Like {\tt push} but add {\tt pc} to the value pushed on to the stack.}
\inst{pushi\_\spic{n}}{}{n}{Push \spic{n} on to the stack. \spic{n} ranges from $-32$ to $31$ inclusive.}
\inst{pushreli\_\spic{n}}{}{addr}{Push ${\tt pc} + {\tt word\_bytes}\times\spic{n}$ on to the stack. \spic{n} ranges from $-64$ to $63$ inclusive.}
\end{description}

\noindent The operand of {\tt pushi} and of {\tt pushreli} is encoded in the instruction opcode; see section~\ref{encoding}.


\subsection{Arithmetic}
\label{arithmetic}

All calculations are made modulo $2^{(8\times{\tt word\_bytes})}$, except
as detailed below.

\nopagebreak
\begin{description}
\inst{neg}{a}{b}{Negate \spic{a}, giving \spic{b}.}
\inst{add}{a b}{c}{Add \spic{a} to \spic{b}, giving the sum \spic{c}.}
\inst{mul}{a b}{c}{Multiply \spic{a} by \spic{b}, giving the product \spic{c}.}
\inst{divmod}{a b}{q r}{Divide \spic{a} by \spic{b}, giving the quotient \spic{q} and remainder \spic{r}. The quotient is rounded towards zero. If \spic{b} is zero, throw error $-8$. If \spic{a} is $-2^{(8\times{\tt word\_bytes} - 1)}$ and \spic{b} is $-1$, throw error $-9$.}
\inst{udivmod}{a b}{q r}{Divide \spic{a} by \spic{b}, giving the quotient \spic{q} and remainder \spic{r}. If \spic{b} is zero, throw error $-8$.}
\end{description}


\subsection{Logic}
\label{logic}

Logic functions:

\nopagebreak
\begin{description}
\inst{not}{a}{b}{\spic{b} is the bitwise complement of \spic{a}.}
\inst{and}{a b}{c}{\spic{c} is the bitwise “and” of \spic{a} with \spic{b}.}
\inst{or}{a b}{c}{\spic{c} is the bitwise inclusive-or of \spic{a} with \spic{b}.}
\inst{xor}{a b}{c}{\spic{c} is the bitwise exclusive-or of \spic{a} with \spic{b}.}
\end{description}


\subsection{Shifts}

\begin{description}
\inst{lshift}{a count}{b}{Shift \spic{a} left by \spic{count} bits, filling vacated bits with zero, giving \spic{b}.}
\inst{rshift}{a count}{b}{Shift \spic{a} right by \spic{count} bits, filling vacated bits with zero, giving \spic{b}.}
\inst{arshift}{a count}{b}{Shift \spic{a} right by \spic{count} bits, filling vacated bits with the most significant bit of \spic{a}, giving \spic{b}.}
\end{description}


\subsection{Comparison}

These instructions compare two numbers on the stack:

\nopagebreak
\begin{description}
\inst{eq}{a b}{flag}{\spic{flag} is~$1$ if \spic{a} is equal to \spic{b}, otherwise $0$.}
\inst{lt}{a b}{flag}{\spic{flag} is~$1$ if \spic{a} is less than \spic{b} when considered as signed numbers, otherwise~$0$.}
\inst{ult}{a b}{flag}{\spic{flag} is~$1$ if \spic{a} is less than \spic{b} when considered as unsigned numbers, otherwise~$0$.}
\end{description}


\subsection{Control}
\label{control}

Unconditional and conditional branches:

\nopagebreak
\begin{description}
\inst{jump}{[addr]}{}{If {\tt ir} is zero, set {\tt pc} to \spic{addr}, which must be aligned, otherwise to ${\tt pc} + \texttt{ir}\times\texttt{word\_bytes}$. Set {\tt ir} to $0$.}

\inst{jumpz}{flag [addr]}{}{If \spic{flag} is $0$ then perform the action of {\tt jump}; otherwise, set {\tt ir} to $0$.}
\end{description}

Subroutine call and return; the quantities called \spic{args} and \spic{rets} are treated as unsigned numbers:

\begin{description}
\instprim{call}{…\x{\spic{args}}\dots\x1 args rets [addr] {\rm →}\\ …rets ret-addr {\rm |} …\x{\spic{args}}\dots\x1}{Let \spic{ret-addr} be the value of {\tt pc}. Perform the action of {\tt jump}. Move \spic{\x{\spic{args}}\dots\x1} from the caller’s stack to the callee’s.}

\instprim{ret}{\pbox[t]{\stackcom}{\qquad\quad …rets ret-addr {\rm |} …\x{\spic{rets}}\dots\x1 → …\x{\spic{rets}}\dots\x1 \\{\rm or\quad} …rets ret-addr {\rm ‖} …\x{\spic{rets}}\dots\x1 → …\x{\spic{rets}}\dots\x1 $0$}}{Pop the current computation stack from the call stack; if the call stack is now empty, pop the current call stack. Set {\tt pc} to \spic{ret-addr}, and move \spic{\x{\spic{rets}}\dots\x1} to the new computation stack. Set {\tt ir} to $0$. If a call stack was popped, push $0$ on to the computation stack.}
\end{description}

Error catchers and raising errors:

\begin{description}
\instprim{catch}{\pbox[t]{\stackcom}{…\x{\spic{args}}\dots\x1 args rets addr →\\ …rets ret-addr {\rm ‖} …\x{\spic{args}}\dots\x1}}{Set {\tt ir} to $0$. Perform the action of {\tt call}, then push a new call stack on to the catch stack and move the top-most computation stack from the previous call stack to the new one.}

\inst{throw}{rets ret-addr {\rm ‖} …n}{n}{Pop the current call stack from the catch stack.}
\end{description}


\subsection{Extra instructions}
\label{extra}

Extra instructions, using the {\tt extra} instruction, offer necessary functionality too rare or slow to deserve a core instruction.

\begin{description}
\inst{extra}{}{}{Perform extra instruction {\tt ir}; if {\tt ir} is not the code of a valid extra instruction, throw error $-1$. The stack effect depends on the extra instruction. Extra instruction $0$ performs instruction fetch (see section~\ref{execution}).}
\end{description}


\subsection{Traps}
\label{trap}

Traps, using the {\tt trap} instruction, are similar to extra instructions, but are intended to be implementable as add-ons to an implementation, rather than as an integrated part of it. Traps may modify the memory and stack, but may not directly change the values of registers.

\begin{description}
\inst{trap}{}{}{Perform trap {\tt ir}; if {\tt ir} is not the code of a valid trap, throw error $-1$. The stack effect depends on the trap. Trap code $-1$ performs instruction fetch (see section~\ref{execution}).}
\end{description}


\subsection{Instruction encoding}
\label{encoding}

Instructions are encoded as bytes, packed into words, which are executed as described in section~\ref{execution}. The bytes have the following internal structure:

\begin{center}
  \begin{bytefield}[endianness=big,bitwidth=2em]{8}
    \bitheader{0-7}\\
    \bitbox{5}{opcode}\bitboxes*{1}{000}\insttbllabel{instruction}\\[1ex]
    \bitbox{5}{$n$}\bitboxes*{1}{100}\insttbllabel{{\tt pushi} $n < 0$}\\[1ex]
    \bitbox{6}{$n$}\bitboxes*{1}{10}\insttbllabel{{\tt pushreli} $n < 0$}\\[1ex]
    \bitbox{6}{$n$}\bitboxes*{1}{01}\insttbllabel{{\tt pushreli} $n\geq 0$}\\[1ex]
    \bitbox{5}{$n$}\bitboxes*{1}{011}\insttbllabel{{\tt pushi} $n\geq 0$}\\[1ex]
    \bitbox{5}{opcode}\bitboxes*{1}{111}\insttbllabel{instruction}
  \end{bytefield}
\end{center}

Table~\ref{opcode000table} lists the opcodes for instructions whose least-significant $3$ bits are $000$, and table~\ref{opcode111table} for $111$. Other instruction opcodes with those endings are invalid. Table~\ref{extraopcodetable} lists the valid extra instruction opcodes.

\begin{table}[htb]
\begin{center}
\begin{tabular}{*{2}{cc}} \toprule
\bf Opcode & \bf Instruction & \bf Opcode & \bf Instruction \\ \midrule
\opcodetbltwo{00}{extra}		{10}{load}
\opcodetbltwo{01}{not}			{11}{store}
\opcodetbltwo{02}{and}			{12}{load1}
\opcodetbltwo{03}{or}			{13}{store1}
\opcodetbltwo{04}{xor}			{14}{load2}
\opcodetbltwo{05}{lshift}		{15}{store2}
\opcodetbltwo{06}{rshift}		{16}{load4}
\smallskip% For some reason needs to be here rather than after the next line
\opcodetbltwo{07}{arshift}		{17}{store4}
\opcodetbltwo{08}{pop}			{18}{push}
\opcodetbltwo{09}{dup}			{19}{pushrel}
\opcodetbltwo{0a}{set}			{1a}{negate}
\opcodetbltwo{0b}{swap}			{1b}{add}
\opcodetbltwo{0c}{jump}			{1c}{mul}
\opcodetbltwo{0d}{jumpz}		{1d}{eq}
\opcodetbltwo{0e}{call}			{1e}{lt}
\opcodetbltwo{0f}{ret}			{1f}{ult}
 \bottomrule
\end{tabular}
\caption{\label{opcode000table}Instruction opcodes for “$000$” instructions}
\end{center}
\end{table}

\begin{table}[htb]
\begin{center}
\begin{tabular}{*{2}{cc}} \toprule
\bf Opcode & \bf Instruction \\ \midrule
\opcodetblone{1f}{trap}
 \bottomrule
\end{tabular}
\caption{\label{opcode111table}Instruction opcodes for “$111$” instructions}
\end{center}
\end{table}

\begin{table}[htb]
\begin{center}
\begin{tabular}{{cc}} \toprule
\bf Opcode & \bf Instruction \\ \midrule
\opcodetblone{1}{divmod}
\opcodetblone{2}{udivmod}
\opcodetblone{3}{catch}
\opcodetblone{4}{throw}
 \bottomrule
\end{tabular}
\caption{\label{extraopcodetable}Extra instruction opcodes}
\end{center}
\end{table}

\section{External interface}

\begin{itemize}
\item Implementations can add extra instructions to provide extra computational primitives and other deeply-integrated facilities, and traps to offer access to system facilities, native libraries and so on; see section~\ref{extra}.
\item Implementations should provide an API to create and run virtual machine code, and to add more traps.
\end{itemize}


\section*{Acknowledgements}

Martin Richards introduced me to Cintcode~\cite{cintweb}, which
kindled my interest in virtual machines, and led to
Beetle~\cite{beetledis} and Mite~\cite{mite0},
of which Mit is a sort of synthesis.
GNU~\emph{lightning}~\cite{lightning} helped inspire me to greater
simplicity, while still aiming for speed. Alistair Turnbull
has for many years been a fount of ideas and criticism for all my work in computation, and lately a staunch collaborator on Mit.

\bibliographystyle{plain}
\bibliography{vm,rrt}


\end{document}

% LocalWords:  Richards's addr Mit catchee catchee’s
