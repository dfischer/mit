% The definition of Mit (self-contained paper)
%
% (c) Reuben Thomas 1993-2020
% This file is in the public domain.

\documentclass[a4paper]{article}
\usepackage[british]{babel}
\usepackage[utf8x]{inputenc}
\usepackage{newpxtext,booktabs,hyperref,bytefield}
\frenchspacing


% Macros for this document

% Font for stack pictures; macro \spic includes italic correction

\newcommand{\spic}[1]{\texttt{\textsl{#1\/}}}

% Common stack items

\newcommand{\x}[1]{\spic{x$_{#1}$}}
\newcommand{\n}[1]{\spic{s$_{#1}$}}
\newcommand{\U}[1]{\spic{u$_{#1}$}}
\newcommand{\noru}[1]{\spic{n$_#1$}}
\newcommand{\addr}[1]{\spic{addr$_{#1}$}}
\newcommand{\aaddr}[1]{\spic{a-addr$_{#1}$}}

% Lay out an instruction definition

% Define the widths of the stack effect and description columns
\newlength{\itemwidth}\itemwidth=\textwidth \advance\itemwidth by -0.1in
\newlength{\instname}\instname=0.8in
\newlength{\stackcom}\stackcom=3.7in

\newcommand{\inst}[3]{\item[]\parbox{\itemwidth}%
{\makebox[\instname][l]{\tt #1}%
\makebox[\stackcom][r]{\spic{#2}}\\[0.5ex]#3}}
\newcommand{\extra}[4]{\item[]\parbox{\itemwidth}%
{\makebox[\instname][l]{\tt #1}\makebox[0pt]{\quad{#2}}%
\makebox[\stackcom][r]{\spic{#3}}\\[0.5ex]#4}}

% Lay out a line of the opcode table

% Tabular hex numerals
% From https://tex.stackexchange.com/questions/543464/tabular-hex-numerals/543466#543466
\usepackage{tokcycle}
\newsavebox\lettwd
\savebox\lettwd{{\ooalign{0\cr a\cr b\cr c\cr d\cr e\cr f}}}
\tokcycleenvironment\tblhex
  {\addcytoks{\makebox[\wd\lettwd]{##1}}}
  {\processtoks{##1}}
  {\addcytoks{##1}}
  {\addcytoks{\makebox[\wd\lettwd]{##1}}}

\newcommand{\opcodetbl}[4]{0x\tblhex#1\endtblhex & {\tt #2} & \tblhex0x#3\endtblhex & {\tt #4} \\}


\title{The Mit virtual machine}
\author{Reuben Thomas}
\date{10th May 2020}

\begin{document}
\maketitle

\section{Introduction}

Mit is a simple virtual machine. It is a stack machine, based on the more complex register machine Mite~\cite{mite0}.
This paper gives a full description of Mit.

Mit is conceptually (and usually in fact) a library, embedded in other programs.


\section{Architecture}

\subsection{Memory model}

The {\bf memory} consists of a possibly discontiguous set of {\bf words}, which are composed of $8$-bit {\bf bytes}. The number of bytes in a word is called {\tt word\_bytes}, and must be $4$ or $8$. An {\bf address} identifies a byte. The address space is flat and linear. Addresses are words. The bytes in a word have contiguous addresses, in big- or little-endian order. The address of a word is that of the byte in it with the lowest address, and is a multiple of {\tt word\_bytes}. An address is {\bf valid} if it identifies a byte of a word in memory.

The choice of byte and word size enable efficient implementation on the vast majority of machine architectures.


\subsection{Registers}
\label{registers}

The registers are word quantities; they are listed, with their functions, in table~\ref{regtable}. The registers are initialised to~$0$.

\begin{table}[htbp]
\begin{center}
\begin{tabular}{cp{3.75in}} \toprule
\bf Register & \bf Function \\
 \midrule
{\tt pc} & The {\tt p}rogram {\tt c}ounter. Points to the next word from which {\tt ir} may be loaded. \\
{\tt ir} & The {\tt i}nstruction {\tt r}egister. Contains instructions to be executed. \\
{\tt stack\_depth} & The number of words in the current stack frame. \\
 \bottomrule
\end{tabular}
\caption{\label{regtable}Registers}
\end{center}
\end{table}


\subsection{Stack}

The stack is a last-in-first-out stack of call frames. The {\bf current frame} is the topmost. Each frame is a stack of words. To {\bf push} a word on to the stack means to add a new word to the top of the current frame, increasing the stack depth by $1$; to {\bf pop} a word means to reduce the stack depth by $1$. Instructions that change the number of words on the stack implicitly pop their arguments and push their results.


\subsection{Execution}
\label{execution}

Execution proceeds as follows:

\begin{tabbing}
\hspace{0.2in}\=rep\=eat\+\+ \\*
let \textbf{opcode} be the least significant byte of {\tt ir} \\*
shift {\tt ir} arithmetically one byte to the right \\*
execute the instruction given by \textbf{opcode}, \\*
\hspace{1em}or throw error $-1$ if the opcode is invalid \-
\end{tabbing}

\noindent {\bf Instruction fetch} means setting {\tt ir} to the word pointed to by {\tt pc} and making {\tt pc} point to the next word. This is done whenever {\tt ir} is $0$ or $-1$. These are encoded respectively as extra instruction $0$ (see section~\ref{extra}) and trap $-1$ (see section~\ref{trap}).

If an error occurs during execution (see section~\ref{errors}),
stack frames are popped until reaching one which contains an error handler.
The error handler, return address and number of return items are popped,
and the error code is pushed.


\subsection{Errors and termination}
\label{errors}

When Mit encounters certain exceptional situations, such as an attempt to access an invalid address, or divide by zero, an {\bf error} may be {\bf thrown}. An {\bf error code} is returned to the caller.

Execution can be terminated explicitly by a {\tt throw} instruction (see section~\ref{extra}), which throws an error.

Error codes are signed numbers. $0$ to $-127$ are reserved for the specification; other error codes may be used by implementations. The meanings of those that may be thrown by Mit are shown in table~\ref{errortable}.

\begin{table}[htbp]
\begin{center}
\begin{tabular}{cp{4in}} \toprule
\bf Code & \bf Meaning \\ \midrule
$0$ & Execution has terminated without error. \\
$-1$ & Invalid opcode (see section~\ref{opcodes}). \\
$-2$ & Stack overflow. \\
$-3$ & Invalid stack read. \\
$-4$ & Invalid stack write. \\
$-5$ & Invalid memory read. \\
$-6$ & Invalid memory write. \\
$-7$ & Address alignment error: thrown when an instruction is given a valid address, but insufficiently aligned. \\
$-8$ & Division by zero attempted (see section~\ref{arithmetic}). \\
$-9$ & Division overflow (see section~\ref{arithmetic}). \\
 \bottomrule
\end{tabular}
\caption{\label{errortable}Errors thrown by Mit}
\end{center}
\end{table}

Errors $-2$ to $-8$ inclusive are optional: an implementation may choose to raise them, or not.


\section{Instruction set}
\label{instset}

The instruction set is listed below,
with the instructions grouped according to function. The
instructions are given in the following format:

\begin{description}
\inst{NAME}{before — after}{Description.}
\end{description}

The first line consists of the name of the instruction on the left, and the stack effect on the right. Underneath is the description.

{\bf Stack effects} are written

\centerline{\tt \spic{before — after}}

\noindent where \spic{before} and \spic{after} are stack pictures showing the items on top of a stack before and after the instruction is executed. An instruction only affects the items shown in its stack effect. A stack item \spic{[x]} in square brackets is optional.

{\bf Stack pictures}
represent the topmost stack items, and are written

\centerline{\spic{i$_n$ i$_{n-1}$\dots i$_2$ i$_1$}}

\noindent where the \spic{i$_k$} are stack items, each of which occupies a word, with \spic{i$_1$} being on top of the stack. The symbols denoting different types of stack item are shown in table~\ref{typetable}.

\begin{table}[htbp]
\begin{center}
\begin{tabular}{cl} \toprule
\bf Symbol & \bf Data type \\ \midrule
\spic{flag} & a Boolean flag, zero for false or non-zero for true \\
\spic{s} & signed number \\
\spic{u} & unsigned number \\
\spic{n} & number (signed or unsigned) \\
\spic{x} & unspecified word \\
\spic{addr} & address \\
\spic{a-addr} & word-aligned address \\
\bottomrule
\end{tabular}
\caption{\label{typetable}Types used in stack effects}
\end{center}
\end{table}

Numbers are represented in two's complement form. \spic{addr} consists of all
valid virtual machine addresses.

Each type may be suffixed by a number in stack pictures; if the same combination
of type and suffix appears more than once in a stack effect, it refers each time to the identical stack item.

Ellipsis is used for indeterminate numbers of specified types of item.


\subsection{Stack manipulation}

These instructions manage the stack:

\nopagebreak
\begin{description}
\inst{pop}{\x{u}\dots\x0 u —}{Remove \spic{\x{u}\dots\x0} from the stack.}
\inst{dup}{\x{u}\dots\x0 u — \x{u}\dots\x0 \x{u}}{Remove \spic{u}. Copy \x{u} to the top of the stack.}
\inst{swap}{\x{u+1}\dots\x0 u — \x0 \x{u}\dots\x1 \x{u+1}}{Exchange the top stack word with the \spic{u+1}th.}
\end{description}


\subsection{Control}
\label{control}

These instructions implement unconditional and conditional branches, and subroutine call and return.

\nopagebreak
\begin{description}
\inst{jump}{[a-addr] —}{If {\tt ir} is $0$, the target address is \spic{a-addr}, otherwise it is $\texttt{ir}\times\texttt{word\_bytes} + {\tt pc}$. Set \spic{pc} to the target address, and {\tt ir} to $0$.}
\inst{jumpz}{flag [a-addr] —}{If \spic{flag} is false then perform the action of {\tt jump}; otherwise, set {\tt ir} to $0$.}
\inst{call}{\x{\U1}\dots\x0 \U1 \U2 [\aaddr1] — \U2 \aaddr2 {\tt ||} \x{\U1}\dots\x0}{Call the initial value of {\tt pc} \aaddr2. Perform the action of {\tt jump}. Move \spic{\x{\U1}\dots\x0} to a new call frame.}
\inst{ret}{[h] \U2 a-addr {\tt ||} \x{\U2}\dots\x0 — \x{u}\dots\x0 [$0$]}{Pop the current stack frame, set {\tt pc} to \spic{a-addr}, and move \spic{\x{\U2}\dots\x0} to the next stack frame. Set {\tt ir} to $0$. If there is an error handler \spic{h} in the next stack frame, discard it and push $0$ on top of the stack.}
\end{description}


\subsection{Memory}

These instructions fetch and store quantities to and from memory.

\begin{description}
\inst{load}{a-addr — x}{Load the word \spic{x} stored at \spic{a-addr}.}
\inst{store}{x a-addr —}{Store \spic{x} at \spic{a-addr}.}
\inst{load1}{addr — x}{Load the byte \spic{x} stored at \spic{addr}. Unused high-order bits are set to zero.}
\inst{store1}{x addr —}{Store the least-significant byte of \spic{x} at \spic{addr}.}
\inst{load2}{addr — x}{Load the $2$-byte quantity \spic{x} stored at \spic{addr}, which must be a multiple of $2$. Unused high-order bits are set to zero.}
\inst{store2}{x addr —}{Store the $2$ least-significant bytes of \spic{x} at \spic{addr}, which must be a multiple of $2$.}
\inst{load4}{addr — x}{Load the $4$-byte quantity \spic{x} stored at \spic{addr}, which must be a multiple of $4$. Any unused high-order bits are set to zero.}
\inst{store4}{x addr —}{Store the $4$ least-significant bytes of \spic{x} at \spic{addr}, which must be a multiple of $4$.}
\end{description}


\subsection{Constants}

\begin{description}
\inst{push}{— n}{Push the word pointed to by {\tt pc} on to the stack, and increment {\tt pc} to point to the following word.}
\inst{pushrel}{— n}{Like {\tt push} but add {\tt pc} to the value pushed on to the stack.}
\inst{pushi\_\spic{n}}{— n}{Push \spic{n} on to the stack. \spic{n} ranges from $-32$ to $31$ inclusive.}
\inst{pushreli\_\spic{n}}{— n}{Push ${\tt pc} + {\tt word\_bytes}\times\spic{n}$ on to the stack. \spic{n} ranges from $-64$ to $63$ inclusive.}
\end{description}

The operand of {\tt pushi} and {\tt pushreli} is encoded in the instruction opcode; see section~\ref{encoding}.


\subsection{Logic}
\label{logic}

Logic functions:

\nopagebreak
\begin{description}
\inst{not}{\x1 — \x2}{Invert all bits of \x1, giving its logical inverse \x2.}
\inst{and}{\x1 \x2 — \x3}{\x3 is the bit-by-bit logical “and” of \x1 with \x2.}
\inst{or}{\x1 \x2 — \x3}{\x3 is the bit-by-bit inclusive-or of \x1 with \x2.}
\inst{xor}{\x1 \x2 — \x3}{\x3 is the bit-by-bit exclusive-or of \x1 with \x2.}
\end{description}


\subsection{Comparison}

These instructions compare two numbers on the stack, returning a flag (for equality, use {\tt xor}; see section~\ref{logic}):

\nopagebreak
\begin{description}
\inst{lt}{\n1 \n2 — flag}{\spic{flag} is $1$ if and only if \n1 is less than \n2.}
\inst{ult}{\U1 \U2 — flag}{\spic{flag} is $1$ if and only if \U1 is less than \U2.}
\end{description}


\subsection{Shifts}

\begin{description}
\inst{lshift}{\x1 u — \x2}{Perform a logical left shift of \spic{u} bit-places on \x1, giving \x2. Put zero into the bits vacated by the shift. If \spic{u} is greater than or equal to the number of bits in a word, \x2 is zero.}
\inst{rshift}{\x1 u — \x2}{Perform a logical right shift of \spic{u} bit-places on \x1, giving \x2. Put zero into the bits vacated by the shift. If \spic{u} is greater than or equal to the number of bits in a word, \x2 is zero.}
\inst{arshift}{\x1 u — \x2}{Perform an arithmetic right shift of \spic{u} bit-places on \x1, giving \x2. Copy the original most-significant bit into the bits vacated by the shift. If \spic{u} is greater than or equal to the number of bits in a word, all the bits of \x2 are the same as the original most-significant bit.}
\end{description}


\subsection{Arithmetic}
\label{arithmetic}

All calculations are made modulo $2^{(8\times{\tt word\_bytes})}$, except
as detailed for certain instructions.

Negation and addition:

\nopagebreak
\begin{description}
\inst{negate}{\n1 — \n2}{Negate \n1, giving \n2.}
\inst{add}{\noru1 \noru2 — \noru3}{Add \noru2 to \noru1, giving the sum \noru3.}
\end{description}

Multiplication and division:

\nopagebreak
\begin{description}
\inst{mul}{\noru1 \noru2 — \noru3}{Multiply \noru1 by \noru2 giving the product \noru3.}
\inst{divmod}{\n1 \n2 — \n3 \n4}{Divide \n1 by \n2 using symmetric division, giving the single-word quotient \n3 and the single-word remainder \n4. The quotient is rounded towards zero. If \n2 is zero, throw error $-8$. If \n1 is $-2^{(8\times{\tt word\_bytes} - 1)}$ and \n2 is $-1$, throw error $-9$.}
\inst{udivmod}{\U1 \U2 — \U3 \U4}{Divide \U1 by \U2, giving the single-word quotient \U3 and the single-word remainder \U4. If \U2 is zero, throw error $-8$.}
\end{description}


\subsection{Extra instructions and traps}
\label{extra}\label{trap}

Extra instructions, using the {\tt extra} instruction, offer necessary functionality too rare or slow to deserve a core instruction. Traps, using the {\tt trap} instruction, are similar, but intended to be implementable as add-ons to an implementation, rather than as an integrated part of it. Traps may modify the memory and stack, but may not directly change the values of registers.

\begin{description}
\inst{extra}{{\rm see below}}{Perform extra instruction {\tt ir}; if {\tt ir} is not the code of a valid extra instruction, throw error $-1$. Extra instruction code $0$ is used for instruction fetch (see section~\ref{execution}).}
\inst{trap}{}{Perform trap {\tt ir}; if {\tt ir} is not the code of a valid trap, throw error $-1$. Trap code $-1$ is used for instruction fetch (see section~\ref{execution}).}
\end{description}

The following extra instructions are defined:

\begin{description}
\extra{stack\_depth}{1}{— u}{Push the value of {\tt stack\_depth} on to the stack.}
\extra{throw}{2}{n —}{Throw error \spic{n}.}
\extra{catch}{3}{\x{u}\dots\x0 \U1 \U2 [\aaddr1] — h \U2 \aaddr2 {\tt ||} \x{u}\dots\x0}{Add an error handler \spic{h} to the current stack frame, then perform the action of {\tt call} (see section~\ref{control}).}
\end{description}


\subsection{Instruction encoding}
\label{encoding}

Instructions are encoded as bytes, packed into words, which are executed as described in section~\ref{execution}. The bytes have the following internal structure:

\begin{center}
  \begin{bytefield}[endianness=big,bitwidth=2em]{8}
    \bitheader{0-7}\\
    \bitbox{5}{instruction}\bitboxes*{1}{000}\\[1ex]
    \bitbox{6}{{\tt pushreli} $n\geq 0$}\bitboxes*{1}{01}\\[1ex]
    \bitbox{5}{{\tt pushi} $n\geq 0$}\bitboxes*{1}{011}\\[1ex]
    \bitbox{5}{{\tt pushi} $n < 0$}\bitboxes*{1}{100}\\[1ex]
    \bitbox{6}{{\tt pushreli} $n < 0$}\bitboxes*{1}{10}\\[1ex]
    \bitbox{5}{instruction}\bitboxes*{1}{111}
  \end{bytefield}
\end{center}


\subsection{Instruction opcodes}
\label{opcodes}

Table~\ref{opcodetable} lists the instruction opcodes. Other instruction opcodes are invalid.

% FIXME: use “Tabular hex numerals”
\begin{table}[htb]
\begin{center}
\begin{tabular}{*{2}{cc}} \toprule
\bf Opcode & \bf Instruction & \bf Opcode & \bf Instruction \\ \midrule
\opcodetbl{0}{extra}			{10}{push}
\opcodetbl{1}{pop}			{11}{pushrel}
\opcodetbl{2}{dup}			{12}{not}
\opcodetbl{3}{swap}			{13}{and}
\opcodetbl{4}{jump}			{14}{or}
\opcodetbl{5}{jumpz}			{15}{xor}
\opcodetbl{6}{call}			{16}{lt}
\smallskip% For some reason needs to be here rather than after the next line
\opcodetbl{7}{ret}			{17}{ult}
\opcodetbl{8}{load}			{18}{lshift}
\opcodetbl{9}{store}			{19}{rshift}
\opcodetbl{a}{load1}			{1a}{arshift}
\opcodetbl{b}{store1}			{1b}{negate}
\opcodetbl{c}{load2}			{1c}{add}
\opcodetbl{d}{store2}			{1d}{mul}
\opcodetbl{e}{load4}			{1e}{divmod}
\opcodetbl{f}{store4}			{1f}{udivmod}
 \bottomrule
\end{tabular}
\caption{\label{opcodetable}Instruction opcodes}
\end{center}
\end{table}


\section{External interface}

\begin{itemize}
\item Implementations should provide an API to create and run virtual machine code, and to add traps.
\item Implementations can add extra instructions to provide extra computational primitives and other deeply-integrated facilities, and traps to offer access to system facilities, native libraries and so on; see section~\ref{extra}.
\end{itemize}


\section*{Acknowledgements}

Martin Richards introduced me to Cintcode~\cite{cintweb}, which
kindled my interest in virtual machines, and led to
Beetle~\cite{beetledis} and Mite~\cite{mite0},
of which Mit is a sort of synthesis.
GNU~\emph{lightning}~\cite{lightning} helped inspire me to greater
simplicity, while still aiming for speed. Alistair Turnbull
has been a great collaborator for all my work on virtual machines.

\bibliographystyle{plain}
\bibliography{vm,rrt}


\end{document}

% LocalWords:  Richards's addr Mit
